\section{Autocompare Implementation}

The implementation of OpenACC autocompare is split between our OpenACC compiler and the OpenACC runtime.
Most of the compiler work was to enable redundant execution of compute constructs on both the host and device.
Our implementation already had the capability of generating code for both host and device and selecting which to execute.
We modified that capability so that instead of selecting whether to launch a device kernel or run the host code, it would do both.
Currently, the CPU code runs sequentially.

The compiler already inserted runtime calls for any explicit and implicit data clauses.
In the example program shown in Listing~\ref{GPUsin}, these calls are inserted at entry to the data construct at line 4 and the exit from that construct at line 11.
In redundant execution mode, these runtime calls follow both the device kernel launch and the host redundant execution.
That allowed us to repurpose those runtime calls to do the data compare.
The compiler sets two flags in the runtime call, one to tell the runtime that it is in redundant mode and should not actually update host values, and a second to tell the runtime to compare values.
\begin{lstlisting}[caption={Sample OpenACC loop}, label=GPUsin,frame=single,style=Cstyle]
void vectorSinGPU(double *A, double * C, uint32_t N)
{
  // Ensure the data is available on the device
  #pragma acc data copyin(A[0:N]) copyout(C[0:N])
  {
    // Compute construct
    #pragma acc kernels loop present(A[0:N],C[0:N]) independent
    for (int i = 0; i < N; i++) {
      C[i] = fsin(A[i]);
    }
  }
}
\end{lstlisting}

By default, our initial implementation will compare values that appear in a \emph{copy} or \emph{copyout} data clauses (explicitly or implicitly), or an \emph{update host} directive.
For the code in Listing~\ref{GPUsin}, the values for \textbt{C} would be compared at the exit of the data construct at line 11, but not the values for \textbt{A} because \textbt{A} is \textbt{copyin} only.
We have also implemented two runtime routines that will compare host and device values for specific variables or arrays, or for all data present on the device.
We have an option to disable the automatic comparisons, for when the user adds those runtime routine calls.



Our OpenACC autocompare implementation uses essentially six routines in the OpenACC runtime.
The runtime routines use the \emph{present table}\cite{wolfe.ashes.17} maintained by the OpenACC runtime.
Our implementation of the \emph{present table} saves the variable or array name, its host address, the corresponding device address, the data type, and the length.
\begin{itemize}
\item The \textbt{uacc_compare_contiguous} routine is given a host array section descriptor.
This descriptor is generated by the compiler for the data directives.
This routine finds contiguous blocks of memory in the array section and calls the \textbt{uacc_compare} routine on each.
\item The \textbt{uacc_compare} routine is given the start and end address of a block of host memory.
This is the workhorse of the autocompare feature:
\begin{itemize}
\item It finds that block of memory in the \emph{present table}, which gives the corresponding device address and the data type.
\item It allocates temporary host memory for that data.
\item It downloads the data from device memory to the temporary memory.
\item It calls the \textbt{pgi_compare} routine to do the actual compare operation.
\end{itemize}
\item The \textbt{pgi_compare} routine is part of the more general CAST runtime.
For each block of memory, it first does a fast \textbt{memcmp} to see if the two blocks are exactly the same.
If not, it does a type-specific compare operation for each element, using the appropriate tolerances.

\item The user-callable \textbt{acc_compare} routine is passed the host address of one array that is also present on the device.
%This routine calls \textbt{uacc_compare} routine for that block of memory as shown in Listing~\ref{GPUsin}.
\item The user-callable \textbt{acc_compare_all} routine has no arguments.
It walks the entire \emph{present table} to find all blocks of memory that are also present on the device, and calls \textbt{uacc_compare} on each block.
\item The \textbt{pgi_compare_error} routine is called with \textbt{pgi_compare} detects an untolerated difference.
This is merely a convenient place for the user to set a breakpoint in a debugger, when trying to find what is going wrong.
\end{itemize}

The user can set various options using the \textbt{PGI_COMPARE} environment variable.
The user can set an \emph{absolute tolerance} or \emph{relative tolerance} for floating point comparisons.
The user can select report options, such as to only report the first $n$ differences, or to skip the first $n$ differences.%, a sample report output is illustrated in Listing~\ref{code:sampleOUT}.
Finally, the user can select the action to take when the report limit is exceeded, to stop execution, continue execution, or to patch the bad results and then continue.
In our implementation, patching the values means updating the device locations with the host values.
The \textbt{PGI_COMPARE} environment variable contains a comma-separated list of options; set Table~\ref{env} for details.
\begin{table}
\begin{center}
\begin{tabular}{ll}
\hline
option & Description \\
% \multicolumn{1}{l}{\rule{0pt}{12pt} \textbt{export PGICOMPARE=option[,option]}
% }&\multicolumn{1}{l}{  Description }\\[2pt]
\hline
\textbt{abs=}\textit{r} & Use $10^{-r}$ as an absolute tolerance \\
\textbt{rel=}\textit{r} & Use $10^{-r}$ as a relative tolerance \\
\textbt{report=}\textit{n} & Report first \textit{n} differences \\
\textbt{skip=}\textit{n}    & Skip the first \textit{n} differences \\
\textbt{patch}   &   Patch erroneous values with correct values \\
\textbt{stop}   &   Stop at after \textbt{report=} differences \\
\textbt{summary}   &   Print a summary of the comparisons and differences found at program exit \\
\hline
\end{tabular}
\end{center}
\caption{Options that can appear in the \textbt{PGI_COMPARE} environment variable.}
\label{env}
\end{table}

An example of the \textbt{summary} option is shown in Listing~\ref{summary}.
This was used to generate the results in the next section.
\begin{lstlisting}[caption={Summary option output for one of the benchmark programs.}, label=summary,frame=single,style=Xstyle]
compared 202 blocks, 3388997632 elements, 13555990528 bytes
1488912980 errors tolerated in 201 blocks
 relative tolerance = 0.000100, rel=4
\end{lstlisting}



%\begin{lstlisting}[caption={Sample report when comparing 20000 elements of float data type and using absolute tolerance of 0.1}, label=code:sampleOUT,frame=single,style=Cstyle]
% Floats at: 0 PASSED the equivalence test: 0.0280372 and 0.0280372 TOL used 0.1
%Floats at: 1 PASSED the equivalence test: 0.16666 and 0.16666 TOL used 0.1
%Floats at: 2 PASSED the equivalence test: 0.119761 and 0.119761 TOL used 0.1
%Floats at: 3 PASSED the equivalence test: 0.575198 and 0.575198 TOL used 0.1
%Floats at: 4 PASSED the equivalence test: 0.356631 and 0.356631 TOL used 0.1
%Floats at: 5 PASSED the equivalence test: 0.894281 and 0.894281 TOL used 0.1
%...
%...
%...
%Floats at: 19994 PASSED the equivalence test: 0.969375 and 0.969375 TOL used 0.1
%Floats at: 19995 PASSED the equivalence test: 0.37513 and 0.37513 TOL used 0.1
%Floats at: 19996 PASSED the equivalence test: 0.264415 and 0.264415 TOL used 0.1
%Floats at: 19997 PASSED the equivalence test: 0.88696 and 0.88696 TOL used 0.1
%Floats at: 19998 PASSED the equivalence test: 0.338411 and 0.338411 TOL used 0.1
%Floats at: 19999 PASSED the equivalence test: 0.389184 and 0.389184 TOL used 0.1
%Number of miss match = 0 first miss match occured at 0 out of 20000 comparisons
%MAX error: 0.000000e+00
%
%\end{lstlisting}
