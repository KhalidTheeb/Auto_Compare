\section{Experiments}

We have measured the overhead of our compare feature to demonstrate its usability.
For the general case, we measure the overhead of the first run, where data values are written to a file, and the test run, where values are compared to the \emph{golden results} file, in both cases measuring execution time compared to the run with no compare overhead.
For the OpenACC autocompare feature, we measure the overhead of the autocompare against the time with no compare overhead.
With OpenACC autocompare, the main overhead is running the program on the CPU as well as the GPU.


We have run our overhead experiments on three of the SPEC ACCEL v1.2 benchmarks.
In each case, the program has an outer time step loop containing the main computation.
For the general compare feature, we inserted a call or calls to \textbt{pgi_compare} inside the loop, to save and compare the state after each time step.
We used the \emph{ref} data set in each case.
We ran the \emph{golden} run on an Intel machine, and \emph{test} runs on the same machine as well as on an IBM POWER 8 machine.
The values shown are:
\begin{itemize}
\item Time (seconds) to run the original program on each machine.
\item Time and overhead (percent slowdown) to run the \emph{golden} run on the Intel machine.
\item Time and overhead to run the \emph{test} run on the each machine.
\item Number of calls to \textbt{pgi_compare}.
\item Size of the \emph{golden results} file (bytes).
\end{itemize}

\begin{table}
\begin{center}
\begin{tabular}{lll}
\hline
Operation/Data Size & 303.ostencil & 357.csp\\
\hline
\textbt{Binary file size (bytes)} & 80,099 & 418,481\\
\textbt{Write data with reporting}   &   0.0 & 0.0\\
\textbt{Reading and comparing with reporting}   &   0.0 & 0.04\\
\textbt{Compare with report} & 0.0 &0.04\\
\textbt{Write data without reporting} & 0.0 & 0.0\\
\textbt{Reading and comparing without reporting} & 0.0 & 0.0\\
\textbt{Compare without report} & 0.0& 0.0\\
\hline
\end{tabular}
\end{center}
\caption{The overhead time incurred due to using pgi-compare on two Spec accel applications }
\label{res1}
\end{table}



For the OpenACC autocompare feature, we ran the same set of programs, comparing the time of the combined \emph{test} run against both the original program on a CPU, and the original program on the same GPU.
The values shown are:
\begin{itemize}
\item Time (seconds) to run the original program on the CPU.
\item Time to run the original program on the GPU.
\item Time to run combined \emph{test} program, and overhead relative to both the CPU and GPU runs.
\item Number of data variables compared.
\item Bytes of data compared.
\end{itemize}

\textbf{Conclusions about the overheads!}
