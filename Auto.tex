\section{Autocompare with OpenACC}

When debugging OpenACC programs targeting GPUs, we have additional problems as well as an important advantage.
The problems include using two different processors in the same application, and managing data traffic between the system memory and the GPU device memory.
The important advantage is that we have two processors, so we can create the reference values on the CPU while the GPU is executing, and we have separate memory for the CPU and GPU, so we have a place to store the reference values and the test values.
Many of the problems that arise when programming GPUs with OpenACC have to do with managing the separate memories (stale data on the GPU or the CPU), or dealing with all the problems of porting to a new processor while part of the program is still running on the old processor.

OpenACC programs have directives to tell the compiler what loops to run in parallel on the GPU (\emph{compute constructs}), and what data to copy to the device and when to update data between device and host (\emph{data constructs}).
Our \emph{OpenACC autocompare} feature uses these directives, and a runtime table that keeps track of the correspondence between host and device data, to select what data to compare.

Ideally, we would compare only the data (and all the data) that was changed in a compute construct.
In small sample programs, this is easy to determine, but in general this is not possible.
Instead, we studied several options for choosing what values to compare between host and device, and at what point to do the compare.
We list these from the least costly to the most.
\begin{itemize}
\item The runtime could compare data when it would otherwise be copied back to the host.
Since the data is already being copied to the host, the only overhead is the actual compare.
However, this would miss intermediate divergent computations for data that lives on the device for a long time.
\item The runtime could compare data that appears in explicit data clauses at the exit of each data or compute construct.
This is more costly, since it compares data even when it will remain on the device, but will distinguish more divergences.
This would also miss data with implicitly-determined data attributes.
This could also miss global data that is referenced only in routines called from a compute construct.
\item The same as above, except including both explicit data clauses and implicitly-determined data.
The same comment about global data applies.
\item The runtime could compare all data 
\end{itemize}


There are errors that \emph{OpenACC autocompare} can not detect.
In particular, since the compare point requires synchronization with the device, any errors due to misplaced or erroneous \emph{async} clauses, or missing \emph{wait} directives or clauses will be hidden.
However, the goal is to detect numeric computational differences between two executions, not to find all errors.
