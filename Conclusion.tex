\section{Conclusion}

The closest prior work to this is the FortranTestGenerator\cite{hovy.iwsehpc.16}, which has some of the same goals.
The FortranTestGenerator is designed to create test cases for specific procedures (Fortran subroutines) by capturing the inputs to the subroutine and generating a driver that will recreate the state of the program before the procedure and then invoke the procedure.
As of publication of the article, validation is not yet automated.
Our design can't be used to create test cases for parts of a program in isolation.
In contrast, our method can test the state of the data at one point, or in many points throughout the execution of a program.

Our design has four advantages.
\begin{itemize}
\item The design requires the user to add runtime API calls or directives only where the data should be saved and compared, minimizing the changes required to the program.
\item The use of directives allows the compiler to implicitly add more information, such as the datatype of the data to be compared, variable name, function name, and file location.
The directives can be maintained in the program, only enabling them with a command line option when testing.
\item The same API calls or directives are used for both the \emph{golden} run, creating the \emph{golden result} file, as for the \emph{test} run, comparing to that file.
The behavior of the API calls or directives are controlled by a runtime environment variable.
\item The procedure can find changes in execution paths as well as changes in data.
\item The OpenACC autocompare allows comparing the GPU code against the CPU code without adding any directives or API calls.
OpenACC programs typically identify data that is copied to the device memory, allowing the CPU code to generate the \emph{golden} results in system memory while the GPU generates \emph{test} results in device memory.
At runtime, the data in the two memories is dynamically compared.
\end{itemize}

With low overhead and flexible compare options, 
