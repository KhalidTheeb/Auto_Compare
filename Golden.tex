\section{Usage and Implementation}

Our compiler supports two ways to use the general compare feature, with runtime API calls or with compiler directives.
The directives simplify the usage quite a bit, since the compiler knows things like the datatype, file and function names, and line number, and can pass these to the compare routines automatically.
In either case, the user introduces API calls or compiler directives at the points in the program where intermediate or final results should be compared.
The directives must be enabled with a command line flag, and will be ignored by other compilers.
This allows a user to leave the directives in place, making their insertion part of the continuing maintenance procedure.
The API calls can be placed in conditionally compiled regions, or will have to be isolated in a testing harness or removed before releasing the software.

The directives use Fortran array notation or the same C array section notation used in OpenACC and OpenMP.
In Fortran, the user can insert a directive like:
\begin{otabbing}
\>\textbt{!$pgi compare(a)}
\end{otabbing}
where \textbt{a} is a Fortran simple variable or array.
The user can have two or more names on a single directive, and can use array sections as well:
\begin{otabbing}
\>\textbt{!$pgi compare(a(:),b(1:100,2:20))}
\end{otabbing}
In C++ or C, the user can insert a pragma like:
\begin{otabbing}
\>\textct{pragma pgi compare(x,a[0:n])}
\end{otabbing}
If \textbt{x} is a scalar, a fixed-size array or C VLA, the compiler knows the size of the array; otherwise, the user must use the array section notation to tell the compiler how big is the dynamically allocated array.
The compiler will interpret the directives, essentially replacing them with runtime calls that handle each array, implicitly including the filename, function name, line number, variable or array name, and datatype information, for more precise error reporting.
There are additional clauses that can appear on the directive, to specify the tolerance to use and how to proceed in case of an error.
The directives are enabled with the \textbt{-Mcompare} compiler flag.

The alternative is to use runtime calls.
In Fortran, the user can insert
\begin{otabbing}
\>\textbt{call pgi_compare(a, n, "real")} \\
\>\textbt{call pgi_compare(b, 20000, "doubleprecision")}
\end{otabbing}
In C++ or C, the user would insert the function:
\begin{otabbing}
\>\textbt{call pgi_compare(x, 1, "float")} \\
\>\textbt{call pgi_compare(a, n, "double")}
\end{otabbing}
This method requires the user to specify the datatype in a string, and won't get the other identifying information automatically, but the compare procedure is otherwise identical.
This method also loses the ability to specify different tolerances and behavior for different arrays.

Whether using directives or API calls, the program will invoke the runtime routine that saves the values to a \emph{golden results} file, or compares the values to that file.
The runtime routine will test the program sequence by comparing \ldots SOMETHING HERE ABOUT PROGRAM SEQUENCE .
The behavior of the runtime routine is controlled by the directive clauses and by the \textbt{PGI_COMPARE} environment variable.
\textbt{PGI_COMPARE} must have a sequence of comma-separated options, and must specify at least either \textbt{create} (to create the \emph{golden results} file) or \textbt{compare}.
Other options are listed in the table in Table~\ref{env}.
\begin{table}
\begin{center}
\begin{tabular}{ll}
\hline
option & Description \\
% \multicolumn{1}{l}{\rule{0pt}{12pt} \textbt{export PGICOMPARE=option[,option]}
% }&\multicolumn{1}{l}{  Description }\\[2pt]
\hline
\textbt{file=}\textit{filename} & Name of \emph{golden results} file \\
\textbt{create}   &   This run creates the \emph{golden results} file \\
\textbt{compare}   &   Compare this run to the \emph{golden results} file \\
\textbt{tolabs=}\textit{r} & Use \textit{r} as an absolute tolerance \\
\textbt{tolrel=}\textit{r} & Use \textit{r} as a relative tolerance \\
\textbt{tolulp=}\textit{n} & Allow \textit{n} differences in ULP \\
\textbt{tolieee} & Test IEEE NaNs\\
\textbt{report=}\textit{n} & Report first \textit{n} differences \\
\textbt{skip=}\textit{n}    & Skip the first \textit{n} differences \\
\textbt{verbose}   & Outputs all details of comparison \\
\textbt{patch}   &   Patch errorneous values with correct values \\
\textbt{stop}   &   Stop at first difference \\
\hline
\end{tabular}
\end{center}
\caption{Options that can appear in the \textbt{PGI_COMPARE} environment variable.}
\label{env}
\end{table}

When comparing integer values, the tolerance is ignored and the test value must match the golden value exactly, or the comparison fails.
When comparing floating point values, the test value will be compared using an absolute tolerance, a relative tolerance, compare ignoring the last ULP digits, or any combination of the three.
If no tolerance is specified, the tolerance is zero.
If any tolerance is exceeded, then the comparison fails.
If the test value is a \emph{NaN} but does not match the golden value, the comparison fails as well.
This comparison scheme was copied from what we use in our nightly and weekly quality assurance tests, where we run over 1,000,000 tests every weekend, plus more during shorter nightly runs.
If a comparison fails, the program can ignore some number of errors or report on them, then either stop, continue execution, or patch the errors and then continue.
