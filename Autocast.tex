% This is based on the LLNCS.DEM the demonstration file of
% the LaTeX macro package from Springer-Verlag
% for Lecture Notes in Computer Science,
% version 2.4 for LaTeX2e as of 16. April 2010
%
% See http://www.springer.com/computer/lncs/lncs+authors?SGWID=0-40209-0-0-0
% for the full guidelines.
%



\documentclass{llncs}
\usepackage{graphicx}
\hyphenation{op-tical net-works semi-conduc-tor}
\usepackage{hyperref}
\usepackage{courier}
\usepackage{color}
\usepackage[scaled]{helvet}
\usepackage{listings}
%\newcounter{codecounter}[subsection]

\definecolor{codegreen}{rgb}{0,0.6,0}
\definecolor{codegray}{rgb}{0.5,0.5,0.5}
\definecolor{codepurple}{rgb}{0.58,0,0.82}
\definecolor{backcolour}{rgb}{0.95,0.95,0.92}
\lstset{numberbychapter=false}

\lstdefinestyle{Cstyle}{
    language=C,
    linewidth=12cm,
    breakatwhitespace=false,
    breaklines=true,
    captionpos=b,
    keepspaces=true,
    backgroundcolor=\color{backcolour},
    commentstyle=\itshape\color{codegreen},
    keywordstyle=\color{blue},
    numberstyle=\tiny\color{codegray},
    stringstyle=\color{purple},
    basicstyle=\footnotesize\ttfamily,
    numbers=left,
    xleftmargin=\parindent,
    framexleftmargin=\parindent,
    numbersep=4pt,
    showspaces=false,
    showstringspaces=false,
    showtabs=false,
    firstnumber=1,
    tabsize=1
}
\lstdefinestyle{Xstyle}{
    language=C,
    linewidth=12cm,
    breakatwhitespace=false,
    breaklines=true,
    captionpos=b,
    keepspaces=true,
    backgroundcolor=\color{backcolour},
    commentstyle=\itshape\color{codegreen},
    keywordstyle=\color{blue},
    numberstyle=\tiny\color{codegray},
    stringstyle=\color{purple},
    basicstyle=\footnotesize\ttfamily,
    %numbers=left,
    xleftmargin=\parindent,
    framexleftmargin=\parindent,
    %numbersep=4pt,
    showspaces=false,
    showstringspaces=false,
    showtabs=false,
    %firstnumber=1,
    tabsize=1
}



% define tabs for otabbing region
\newcommand{\otabs}{\hspace*{8mm}\=\hspace*{8mm}\=\hspace{8mm}\=}
% otabbing for creating examples
\newenvironment{otabbing}[1][\otabs]
	{\begin{tabbing}#1\kill}
	{\end{tabbing}}
% typewriter font bold, like textbt, but with the C '#' before pragma
\newcommand{\textct}[1]{\texttt{\textbf{\#\detokenize{#1}}}}
% typewriter font bold after detokenizing, so we can use underscores
\newcommand{\textbt}[1]{\texttt{\textbf{\detokenize{#1}}}}
% typewriter font bold without detokenizing, so we can use \{ and \}
\newcommand{\textat}[1]{\texttt{\textbf{#1}}}

\begin{document}

\title{Automatic Testing of OpenACC Applications}
%
\titlerunning{Automatic Testing of OpenACC}  % abbreviated title (for running head)
%                                     also used for the TOC unless
%                                     \toctitle is used
%
\author{Khalid Ahmad\inst{1} \and Michael Wolfe\inst{2}}
%
\authorrunning{Khalid Ahmad et al.} % abbreviated author list (for running head)
%
%%%% list of authors for the TOC (use if author list has to be modified)
\tocauthor{Khalid Ahmad, Michael Wolfe}
%
\institute{University of Utah,
Salt Lake City, UT, USA\\
\email{Khalid@cs.utah.edu}
\and
NVIDIA/PGI, Beaverton, OR, USA\\ \email{mwolfe@nvidia.com}}

\maketitle              % typeset the title of the contribution

\begin{abstract}
CAST (Compiler-Assisted Software Testing) is a feature in our compiler and runtime to help users automate testing high performance numerical programs.
CAST normally works by running a known working version of a program and saving intermediate results to a reference file, then running a test version of a program and comparing the intermediate results against the reference file.
Here, we describe the special case of using CAST on OpenACC programs running on a GPU.
Instead of saving and comparing against a saved reference file, the compiler generates code to run each compute region on both the host CPU and the GPU.
The values computed on the host and GPU are then compared, using OpenACC data directives and clauses to decide what data to compare.
\keywords{Program testing}
\end{abstract}
%



\section{Introduction}

Porting an application to another processor, or adding parallel algorithms, or even enabling new optimizations can create challenges for testing.
The goal may be to get higher performance, but the programmer must also test that the computed answers are still accurate.
While essentially all processors use the same IEEE floating point representation, not all will support the same features, such as FMA (fused multiple-add) operations.
Different processors, different algorithms, different programs implementing the same algorithm, or different compiler optimizations on the same program can generate a different sequence of operations, producing different floating point roundoff behavior.
To validate an updated program may require identifying at which point in the program the results start to diverge and determining if the divergence is significant.
We are developing a feature in the PGI compilers and runtime called PGI Compiler-Assisted Software Testing or PCAST.
The programmer adds PCAST runtime calls or directives to a working program to save a sequence of intermediate results to a reference file.
The same runtime calls or directives in the updated or ported program will then compare the sequence of intermediate results to those in the reference file.

Porting an application to an accelerator, like a GPU, has even more challenges for testing.
With an accelerator, some (perhaps most) of the computations are done on a processor with a different instruction set, a different set of floating point units, and different numerical libraries.
It's enough of a problem to test that a port of an application to a new processor is correct, or that enabling a new optimization still produces correct answers, but adding the complexity of using an accelerator for part of the computation exacerbates that even more.
Here, we describe the \emph{OpenACC autocompare} feature of PCAST for the special case of testing an OpenACC~\cite{openacc.16} program that targets GPU parallel execution.
The goal is to determine just where the computation starts to go bad.
A difference may be due to the same problems that arise when porting to any new processor, or changing optimizations, or running in parallel.
However, the difference may also be due to the unique behavior of accelerated applications, such as stale data on the device because of missing data \emph{update} operations.

Our PCAST autocompare implementation executes the parallel kernels on the CPU as well as on the GPU in a single run, and then compares the two results.
The user can set options such as floating point tolerances, choosing what to report, and how to proceed if there are differences.
For instance, the user can choose to stop the program after $n$ differences or to replace the bad results with the known good values and continue.

The next section describes some of the problems that arise when porting or optimizing an application, the specific problems of testing application ports to a GPU, and the usage cases that the OpenACC autocompare feature is intended to support.
Section 3 describes the OpenACC autocompare feature in more detail, including how to use it and other details.
Section 4 gives some details of the implementation.
Section 5 gives measurements of the overhead of the autocompare feature.
Section 6 describes related work. 
Section 7 describes work in progress, and 
the final section summarizes the motivation behind the autocompare feature.




\section{Testing a GPU Port of a Numerical Application}

The general problem is to test whether porting a numerical application to a GPU generates different answers.
Since these are typically floating point applications, the meaning of \emph{different} depends on the precision needed.
Since all processors now use IEEE floating point arithmetic\cite{goldberg.cs.91}, many programmers now expect that moving a program to another processor or to a GPU should produce exactly the same answer.
However, different compilers and libraries (even on the same processor) can give different results, for many well-known reasons, including:
\begin{itemize}
\item Different optimizations (changing a/5.0 into a*0.2, or using fused multiply-add instructions in different places), or
\item Presence or absence of FMA operations on the host or GPU, or different FMA association on host and GPU ($(a*b)+(c*d)$ treated as $\textit{fma}(a,b,c*d)$ or $\textit{fma}(c,d,a*b)$).
\item Different transcendentals (different implementations of exp, sin, sqrt), or
\item Parallel execution on the GPU can result in different order of operations, specifically for reductions (sums).
\end{itemize}
A good testing scheme must test for significant differences, but allow for insignificant differences, as long as the final result is accurate.

The test should determine that not only that there are differences, but also identify where the differences are introduced.
Thus, the testing process should compare intermediate results as well as the final result.
There is also the problem of what to do when an error is found.
It could be treated the same as a floating point exception, giving an error message and terminating the program.
Another option is to report the error and continue, perhaps allowing identification of more errors, with perhaps a limit on the number of errors reported.
A third option is to replace the erroneous values with the known correct values before continuing.

When debugging OpenACC programs targeting GPUs, we have additional problems as well as an important advantage.
The problems include using two different processors in the same application, and managing data traffic between the system memory and the GPU device memory.
The important advantage is that we have two processors, so we can create the reference values on the CPU while the GPU is executing, and we have separate memory for the CPU and GPU, so we have a place to store the reference values and the test values.
Many of the problems that arise when programming GPUs with OpenACC have to do with managing the separate memories (stale data on the GPU or the CPU), or dealing with all the problems of porting to a new processor while part of the program is still running on the old processor.

OpenACC programs have directives to tell the compiler what loops to run in parallel on the GPU (\emph{compute constructs}), and what data to copy to the device and when to update data between device and host (\emph{data constructs}).
Our \emph{OpenACC autocompare} feature uses these directives, along with a table in the OpenACC runtime that keeps track of the correspondence between host and device data, to select what data to compare.

\section{Autocompare with OpenACC}

The PCAST OpenACC autocompare feature works by executing each OpenACC compute kernel on both the GPU and the host CPU.
This keeps the values in system memory in sync with the values in GPU device memory, assuming the same computation is done correctly on each. Figure 1 illustrates the flow of regular OpenACC program with and without turning on the PCAST autocomapre option. The computationally intensive part of the illustration can consist of more than one single kernel.

\begin{figure*}[t]
    \centering
    \includegraphics [width=1\linewidth] {flow_chart.pdf}
    \caption{An overview of the autocompare functionality.}
    \label{fig:cfg_figure}
\end{figure*}


The next, equally important step is to compare the computed values on the CPU with those on the GPU.
We assume that the CPU values are the reference values, and the GPU values are the ones being tested.
Ideally, we would compare only the data that was changed in a compute construct.
In small sample programs, this is easy to automatically determine, but in general this is not feasible.
Instead, we studied several options for choosing what values to compare between host and device, and at what point to do the compare.
\begin{enumerate}
\item The runtime could compare all the data in GPU memory to the corresponding CPU memory after each kernel launch.
This is feasible, but likely to be prohibitively expensive.
Large applications can fill the 16GB device memory (on an NVIDIA Pascal GPU), and bringing that much data back and comparing it after each kernel launch would be very expensive.
However, it would certainly be able to identify the specific kernel where results start to diverge.

\item The runtime could compare data only at the end of a compute construct, and only data that is either in an explicit data clause or is explicitly referenced in the construct.
This is less costly than comparing all data, but it could miss updated global data that is modified only in routines called from the compute construct.

\item The runtime could compare data only at the end of a compute construct, and only that data in an explicit data clause on that compute construct (not data implicitly copied to the GPU, or in a clause for an outer data construct).
This is even less costly, and only compares data that the programmer thought important enough to include in a data clause.

\item The runtime could compare data only at the end of a data region, and only the data in explicit data clauses.
This is likely to be less costly because a data region typically contains many compute constructs, such as an outer loop that contains a compute construct.
However it is less precise about identifying which particular compute construct caused the divergence.

\item The runtime could compare data when it would otherwise be copied back to the host.
This would be at the end of a compute or data construct, or at an OpenACC \emph{update} directive.
Since the data is already being copied to the host, the only overhead is the actual compare.
This method would be even less precise about identifying where the divergent computations occurred.

\item Finally, the runtime could leave the choice to the user.
This would allow the user to insert a runtime call or a directive that tells the runtime when to compare data, and what data to compare.
We considered two options: one where the user chose one or more variables or arrays to compare, and a second where the user asks the runtime to compare all variables and arrays present in device memory to the corresponding host memory locations.
\end{enumerate}

In all cases, since the same computations are done on the CPU as well as the GPU, the OpenACC runtime must allow for this and not do any actual data downloads from the device memory to host memory.
This means \emph{update host} directives and \emph{copyout} actions from data directives should not update the host values.

There are errors that the PCAST OpenACC autocompare feature can not detect.
In particular, since the compare point requires synchronization with the device, any errors due to misplaced or erroneous \emph{async} clauses, or missing \emph{wait} directives or clauses will be hidden.
Also, OpenACC has features that allow different computations on the host as on the device, to allow for different algorithmic formulations that are more appropriate for each processor.
Such a feature can allow for programming mistakes that are hard to detect.
Again, the goal is to detect numeric computational differences between two executions, not to find all errors.

\section{Autocompare Implementation}

The OpenACC autocompare feature is enabled with a command line option.
When enabled, the compiler generates code for each parallel construct for both the CPU and the GPU.
It launches the kernel on the GPU, then executes the corresponding code on the CPU.
OpenACC programs have directives to tell the compiler what loops to run in parallel on the GPU (\emph{compute constructs}), and what data to copy to the device and when to update data between device and host (\emph{data constructs}).
Our \emph{OpenACC autocompare} feature uses these directives, along with a table in the OpenACC runtime that keeps track of the correspondence between host and device data, to select what data to compare.

The implementation of OpenACC autocompare is split between our OpenACC compiler and the OpenACC runtime.
Most of the compiler work was to enable redundant execution of compute constructs on both the host and device.
Our implementation already had the capability of generating code for both host and device and selecting which to execute.
We modified that capability so that instead of selecting whether to launch a device kernel or run the host code, it would do both.
Currently, the CPU code runs sequentially.

The compiler already inserted runtime calls for any explicit and implicit data clauses.
In the example program shown in Listing~\ref{GPUsin}, these calls are inserted at entry to the data construct at line 4 and the exit from that construct at line 11.
In redundant execution mode, these runtime calls follow both the device kernel launch and the host redundant execution.
That allowed us to repurpose those runtime calls to do the data compare.
The compiler sets two flags in the runtime call, one to tell the runtime that it is in redundant mode and should not actually update host values, and a second to tell the runtime to compare values.
\begin{lstlisting}[caption={Sample OpenACC loop}, label=GPUsin,frame=single,style=Cstyle]
void vectorSinGPU(double *A, double * C, uint32_t N)
{
  // Ensure the data is available on the device
  #pragma acc data copyin(A[0:N]) copyout(C[0:N])
  {
    // Compute construct
    #pragma acc kernels loop present(A[0:N],C[0:N]) independent
    for (int i = 0; i < N; i++) {
      C[i] = fsin(A[i]);
    }
  }
}
\end{lstlisting}

By default, our initial implementation will compare values that appear in a \emph{copy} or \emph{copyout} data clauses (explicitly or implicitly), or an \emph{update host} directive.
For the code in Listing~\ref{GPUsin}, the values for \textbt{C} would be compared at the exit of the data construct at line 11, but not the values for \textbt{A} because \textbt{A} is \textbt{copyin} only.
We have also implemented two runtime routines that will compare host and device values for specific variables or arrays, or for all data present on the device.
We have an option to enable redundant execution but disable the automatic comparisons, for when the user adds those runtime routine calls.



Our OpenACC autocompare implementation uses essentially six routines in the OpenACC runtime.
The runtime routines use the \emph{present table}\cite{wolfe.ashes.17} maintained by the OpenACC runtime.
Our implementation of the \emph{present table} saves the variable or array name, its host address, the corresponding device address, the data type, and the length.
\begin{itemize}
\item The \textbt{uacc_compare_contiguous} routine is given a host array section descriptor.
This descriptor is generated by the compiler for the data directives.
This routine finds contiguous blocks of memory in the array section and calls the \textbt{uacc_compare} routine on each.
\item The \textbt{uacc_compare} routine is given the start and end address of a block of host memory.
This is the workhorse of the autocompare feature:
\begin{itemize}
\item It finds that block of memory in the \emph{present table}, which gives the corresponding device address and the data type.
\item It allocates temporary host memory for that data.
\item It downloads the data from device memory to the temporary memory.
\item It calls the \textbt{pgi_compare} routine to do the actual compare operation.
\end{itemize}
\item The \textbt{pgi_compare} routine is part of the more general CAST runtime.
For each block of memory, it first does a fast \textbt{memcmp} to see if the two blocks are exactly the same.
If not, it does a type-specific compare operation for each element, using the appropriate tolerances.

\item The user-callable \textbt{acc_compare} routine is passed the host address of one array that is also present on the device.
This routine calls \textbt{uacc_compare} routine for that block of memory. % as shown in Listing~\ref{GPUsin}.
\item The user-callable \textbt{acc_compare_all} routine has no arguments.
It walks the entire \emph{present table} to find all blocks of memory that are also present on the device, and calls \textbt{uacc_compare} on each block.
\item The \textbt{pgi_compare_error} routine is called when \textbt{pgi_compare} detects an untolerated difference.
This is merely a convenient place for the user to set a breakpoint in a debugger, when trying to find what is going wrong.
\end{itemize}

The user can set various options using the \textbt{PGI_COMPARE} environment variable.
The user can set an \emph{absolute tolerance} or \emph{relative tolerance} for floating point comparisons.
The user can select report options, such as to only report the first $n$ differences, or to skip the first $n$ differences. %, a sample report output is illustrated in Listing~\ref{code:sampleOUT}.
Finally, the user can select the action to take when the report limit is exceeded: to stop execution, continue execution, or to patch the bad results and then continue.
In our implementation, patching the values means updating the device locations with the host values.
The \textbt{PGI_COMPARE} environment variable contains a comma-separated list of options; set Table~\ref{env} for details.
\begin{table}
\begin{center}
\begin{tabular}{ll}
\hline
option & Description \\
% \multicolumn{1}{l}{\rule{0pt}{12pt} \textbt{export PGICOMPARE=option[,option]}
% }&\multicolumn{1}{l}{  Description }\\[2pt]
\hline
\textbt{abs=}\textit{r} & Use $10^{-r}$ as an absolute tolerance \\
\textbt{rel=}\textit{r} & Use $10^{-r}$ as a relative tolerance \\
\textbt{report=}\textit{n} & Report first \textit{n} differences \\
\textbt{skip=}\textit{n}    & Skip the first \textit{n} differences \\
\textbt{patch}   &   Patch erroneous values with correct values \\
\textbt{stop}   &   Stop at after \textbt{report=} differences \\
\textbt{summary}   &   Print a summary of the comparisons and differences found at program exit \\
\hline
\end{tabular}
\end{center}
\caption{Options that can appear in the \textbt{PGI_COMPARE} environment variable.}
\label{env}
\end{table}

An example of the \textbt{summary} option is shown in Listing~\ref{summary}.
This feature was used to generate the results in the next section.
\begin{lstlisting}[caption={Summary option output for one of the benchmark programs.}, label=summary,frame=single,style=Xstyle]
compared 202 blocks, 3388997632 elements, 13555990528 bytes
1488912980 errors tolerated in 201 blocks
 relative tolerance = 0.000100, rel=4
\end{lstlisting}



%\begin{lstlisting}[caption={Sample report when comparing 20000 elements of float data type and using absolute tolerance of 0.1}, label=code:sampleOUT,frame=single,style=Cstyle]
% Floats at: 0 PASSED the equivalence test: 0.0280372 and 0.0280372 TOL used 0.1
%Floats at: 1 PASSED the equivalence test: 0.16666 and 0.16666 TOL used 0.1
%Floats at: 2 PASSED the equivalence test: 0.119761 and 0.119761 TOL used 0.1
%Floats at: 3 PASSED the equivalence test: 0.575198 and 0.575198 TOL used 0.1
%Floats at: 4 PASSED the equivalence test: 0.356631 and 0.356631 TOL used 0.1
%Floats at: 5 PASSED the equivalence test: 0.894281 and 0.894281 TOL used 0.1
%...
%...
%...
%Floats at: 19994 PASSED the equivalence test: 0.969375 and 0.969375 TOL used 0.1
%Floats at: 19995 PASSED the equivalence test: 0.37513 and 0.37513 TOL used 0.1
%Floats at: 19996 PASSED the equivalence test: 0.264415 and 0.264415 TOL used 0.1
%Floats at: 19997 PASSED the equivalence test: 0.88696 and 0.88696 TOL used 0.1
%Floats at: 19998 PASSED the equivalence test: 0.338411 and 0.338411 TOL used 0.1
%Floats at: 19999 PASSED the equivalence test: 0.389184 and 0.389184 TOL used 0.1
%Number of miss match = 0 first miss match occured at 0 out of 20000 comparisons
%MAX error: 0.000000e+00
%
%\end{lstlisting}

\section{Experiments}

We have measured the overhead of the PCAST OpenACC autocompare implementation to demonstrate its usability.
We used ten of the SPEC ACCEL v1.2 benchmarks, using the \emph{test} dataset.
In each case, the program has an outer time step loop containing the main computation.
The times shown are in seconds, and these are officially SPEC \emph{estimates}, since they were not run in the SPEC harness.
%The host machine was a 6-core Intel Haswell (core i7-5820K) with a 3.30GHz clock, with an NVIDIA Tesla Kepler K40c GPU.
The host machine was a dual socket 16-core Intel Haswell (E5-2698 Xeon, 32-cores total) with a 2.30GHz clock, with an NVIDIA Tesla Pascal P100 GPU.
We used the default autocompare options, but set a relative tolerance.
The execution times are in seconds, measured with /usr/bin/time.
%The values shown in Table~\ref{res1} and Figure~\ref{fig:sle_figure} are:
The values shown in Table~\ref{res1} are:
\begin{itemize}
\item Time to run the test data set sequentially on the CPU.
\item Time to run the test data set in parallel on the GPU.
\item Time to run the test data set redundantly on both CPU and GPU without the autocompare feature enabled.
\item Time to copy the data from GPU to CPU before comparing, measured by nvprof.
\item Time to run the test data set redundantly on both CPU and GPU using the autocompare feature.
\end{itemize}


%\begin{table}
%\begin{center}
%\begin{tabular}{rrrrl}
%\hline
% 303.ostencil & 304.olbm & 363.swim & \\
%\hline
% 3.40s &  3.27s & 1.80s & CPU time (sequential) \\
% 3.27s &  2.79s & 1.72s & GPU time \\
% 8.80s &  3.97s & 1.88s & redundant execution on CPU and GPU \\
% 1.80s &  1.69s & 0.09s & CPU to GPU data copy time \\
%64.38s & 23.95s & 2.80s & autocompare time \\
%202 & 61 & 258 & variables and arrays compared \\
%3,388,997,632 & 1,586,800,000 & 67,897,602 & values compared \\
%\hline
%\end{tabular}
%\end{center}
%\caption{Results showing overhead of OpenACC autocompare.}
%\label{res1}
%\end{table}

\begin{table}
\begin{center}
\begin{tabular}{|l|c|c|c|c|c|c|c|c|c|}
\hline

Benchmark & \parbox[c]{1.9 cm}{\centering CPU time \\(sequential)} &  \parbox[c]{1 cm}{\centering GPU \\ time} & \parbox[c]{3.5 cm}{\centering Redundant execution \\ on CPU and GPU} & \parbox[c]{2.75 cm}{\centering CPU to GPU \\  data copy time} &  \parbox[c]{2 cm}{\centering Autocompare \\   time} \\



\hline
%303.ostencil  & 3.40s & 3.25s & 4.34s & 1.14s & 17.58s \\
%304.olbm      & 2.44s & 1.54s & 3.20s & 0.96s & 22.32s \\
%363.swim      & 1.64s & 1.13s & 1.15s & 0.04s & 2.06s \\

ostencil  & 3.51 & 1.82 & 4.22 & 1.03 & 17.19\\
olbm      & 2.19 & 1.30 & 3.03 & 0.96 & 19.09 \\
omriq     & 1.49 & 0.88 & 2.05 & 0.03 & 2.08\\
palm      & 2.75 & 1.45 & 3.75 & 0.50 & 15.75\\
ep        & 2.50 & 0.98 & 3.19 & 0.11 & 3.21\\
miniGhost & 0.87 & 1.07 & 1.69 & 1.23 & 13.17\\
cg        & 62.98 & 28.74 & 64.86 & 0.28 & 68.43\\
csp       & 2.78 & 1.20	& 3.64 & 26.69 & 309.99\\
ilbdc     & 160.62 & 2.10 & 160.39 & 27.41 & 615.26\\
bt        & 5.92 & 1.27 & 7.27 & 9.13 & 119.28\\

            
\hline
\end{tabular}
\end{center}
\caption{Results showing overhead of the PCAST OpenACC autocompare feature.}
\label{res1}
\end{table}


In Table~\ref{res2} 
we show some statistics about the benchmarks we used, such as:
\begin{itemize}
\item Number of variables or arrays compared.
\item Number of data values compared.
\item Number of variables or arrays that had some differences.
\item Number of data values that were different.
\end{itemize}




\begin{table}
\begin{center}
\begin{tabular}{|l|c|c|c|c|c|c|c|c|c|}
\hline

Benchmark &  \parbox[c]{3 cm}{\centering Variables and \\ arrays compared} &  \parbox[c]{2 cm}{\centering Values \\ compared} &  \parbox[c]{3 cm}{\centering variables and arrays \\     with differences} & \parbox[c]{2 cm}{\centering Differences \\      tolerated}\\



\hline
%303.ostencil & 202 & 3,388,997,632 & 0 & 0 \\
%304.olbm     & 61 & 1,586,800,000 & 59 & 520,634,266 \\
%363.swim     & 259 & 67,897,602 & 259 & 22,336,658 \\

ostencil  & 202 & 3388997632 & 0 & 0\\
olbm      & 61 & 586800000 & 59 & 520634266\\
omriq     & 3 & 68608 & 2 & 53240 \\
palm      & 31244 & 1532482935 & 14784 & 374679922\\
ep        & 4 & 13 & 2 & 2 \\
miniGhost & 2506 & 1844059545 & 175 & 175\\
cg        & 186 & 621600195 & 168 & 4858272\\
csp       & 4057 & 40132155677	& 3897 & 5693059\\
ilbdc     & 3001 & 53818895200 & 2000 & 35305830600 \\
bt        & 5036 & 15041440200 & 4798 & 38931891 \\

            
\hline
\end{tabular}
\end{center}
\caption{Results showing number of variables and values processed by the PCAST OpenACC autocompare.}
\label{res2}
\end{table}




Figure~\ref{fig:sle_figure} breaks down the time spent in the autocompare run into:
\begin{itemize}
\item Compute Time: the max of the CPU time and GPU time from Table~\ref{res1}.
\item Redundancy Overhead: the difference between the redundant execution time from Table~\ref{res1} and Compute Time.
\item Download Time: the time spent downloading data as measured using nvprof.
\item Compare Time: the difference between autocompare time from Table~\ref{res1} and the sum of the above three times.
\end{itemize}

%\begin{figure*}[t]
%    \centering
%    \includegraphics [width=1\linewidth] {Table1.pdf}
%    \caption{Results showing overhead of the PCAST OpenACC autocompare feature.}
%    \label{fig:sle_figure}
%\end{figure*}

\begin{figure*}[t]
    \centering
    \includegraphics [width=1\linewidth] {npic3.pdf}
    \caption{Results showing overhead of the PCAST OpenACC autocompare feature.}
    \label{fig:sle_figure}
\end{figure*}


The costs of the autocompare feature are running the compute region on both CPU and GPU, and downloading and comparing the values.
The cost of redundant execution is less than the sum of the CPU and GPU times, because the GPU code executes asynchronously while the CPU executes the corresponding code.
Since this is a feature used during code development and debugging, we consider this to be relatively low overhead.
The cost of doing the many floating point comparisons is significant, and appears to be directly related to the number of data items compared, and unrelated to the number of arrays or variables being compared.
However, using this feature to find where a GPU computation diverges moves the cost from the programmer to the computer, so it could be invaluable regardless of the overhead. An interesting observation we came across was that on all benchmarks we tested it roughly took one second to compare a gigabyte of data as can be seen in Figure~\ref{fig:gbps_figure}.
This can be used as a rough estimate of how long it will take to test a code for correctness as long as the user knows the size of the data set being used.

\begin{figure*}[t]
    \centering
    \includegraphics [width=1\linewidth] {GBperSec.pdf}
    \caption{Results showing amount of data the autocompare feature process per unit time.}
    \label{fig:gbps_figure}
\end{figure*}

One side note: the \emph{test} datasets used here are relatively small.
Even so, we had to set the relative tolerance to avoid the comparisons detecting differences, mostly due to different summation accumulation order.
Surprisingly, those differences propagated to about $\frac{1}{3}$ of the results in two of the ten benchmarks that we show here.
This seems to imply that the cost of computing and comparing a quick checksum or signature before downloading and comparing all the data would frequently have little or no benefit.

\section{Related Work}


The PCAST OpenACC autocompare feature is similar in some respects to redundant execution strategies, which are typically used to detect failing hardware or erroneous software.
The Tandem NonStop computers~\cite{bartlett.tandem.86} implemented redundant execution on identical hardware with automatic checking; the system could detect faulty hardware and fail-over to another processor.
The NASA Space Shuttle carried five computers~\cite{fraser.astro.74}; four of these comprised the primary system and ran identical software with a voting protocol to detect a failing computer.
If the four primary system computers could not determine a correct result, the fifth backup system was enabled for ascent and landing.
The major difference is the autocompare feature assumes that the CPU execution is correct, and it compares the GPU computations to those assumed correct results.

The Cray Comparative Debugger (CCDB)~\cite{derose.sc.15} allows a programmer to launch two versions of a program, such as a CPU-only version and a GPU-accelerated version, and to inspect and compare values between the two versions.
The programmer can also add breakpoints and have the debugger compare specific values between the two program versions when the two versions reach the breakpoint.
This is perhaps the most aggressive approach to allow value comparisons between two running executions and allows a user to inspect the program when the values diverge, although the comparisons themselves are not performed automatically.

Research using the OpenARC compiler framework~\cite{lee.hpdc.14} has explored several strategies for debugging OpenACC programs~\cite{lee.ipdps.14}.
One of these strategies is a mechanism very like the PCAST autocompare feature, where the compiler generates device code and host sequential code for specific compute regions.
The user selects specific compute regions to verify, and the rest of the program is executed sequentially by the host CPU, including other compute regions.
All the data is copied from the system memory to the GPU before those selected kernels, and all modified data is copied back and compared afterward.
That work allows more fine grain control of which compute regions to compare, but does not allow for a unified framework to run the whole program and compare data after each compute region.

\section{Future Work}

We are considering future work on the autocompare feature, including:
\begin{itemize}
\item Optimizing the comparison operation, since this seems to be the bottleneck for the autocompare feature.
\item Ways to reduce the number of values being compared, to reduce the runtime cost.
\item Running the compare itself in parallel, and perhaps running the compare code on the GPU itself.
\item Some way to isolate regions of the program that are being tested, so as to not require redundant execution and data compares throughout the whole execution.
\item Running the host code in parallel as well, for performance, though this assumes that host parallel execution is as accurate and precise as the sequential execution.
\item Adding support for arrays of structs or arrays of derived types, where each field would have a type-specific compare.
\item Adding support for nested data structures, where a struct memory is a pointer to another array.
The compare function would recurse to compare the nested structure as well.
\end{itemize}

%


%\section{Obstacles for Unit Tests in HPC}

%\section {What Problems are addressed}
%\subsection{comparing results between a known good version (golden) and a version being tested}
%  - deciding what values to save / compare
%  - deciding when to save / compare values
%  - deciding how to do the compare, esp. for floating point
%\subsection{Usage Scenarios}
%  - user driven, save golden values to a file, compare test version against those
%    using API calls or directives
%    user decides what values to save/compare, when to compare
%  - automatic online compare, compare CPU vs GPU computations
%    can either compare after each kernel launch, or
%    compare all values present on device against golden host values

%\section{Implementation Details}
%   compare options - tolerance, IEEE comparisons
%   where in the runtime it is implemented (?)
%   challenges: Fortran vs C, gcc-specific header file functionality
%   saving a golden file, file block headers to compare execution sequence

%\section{Experimentation}
%   cost overhead of compare
%   cost overhead of autocompare
%   what kinds of differences we find
%
%\section{Other Uses of a Compare Feature}
%   compare different algorithms
%   compare compiler optimizations
%   compare different hosts

\bibliographystyle{IEEEtran}
\bibliography{main}



\end{document}
