% This is based on the LLNCS.DEM the demonstration file of
% the LaTeX macro package from Springer-Verlag
% for Lecture Notes in Computer Science,
% version 2.4 for LaTeX2e as of 16. April 2010
%
% See http://www.springer.com/computer/lncs/lncs+authors?SGWID=0-40209-0-0-0
% for the full guidelines.
%
\documentclass{llncs}

\begin{document}

\title{Compiler Support for Testing Numerical HPC Applications}
%
\titlerunning{Hamiltonian Mechanics}  % abbreviated title (for running head)
%                                     also used for the TOC unless
%                                     \toctitle is used
%
\author{Khalid Ahmad\inst{1} \and Michael Wolfe\inst{2}}
%
\authorrunning{Khalid Ahmad et al.} % abbreviated author list (for running head)
%
%%%% list of authors for the TOC (use if author list has to be modified)
\tocauthor{Khalid Ahmad, Roger Temam}
%
\institute{University of Utah,
, Salt Lake City, Utah, USA\\
\email{Khalid@cs.utah.edu}
\and
PGI Compilers and tools; an NVIDIA brand, Portland, OR, USA}

\maketitle              % typeset the title of the contribution

\begin{abstract}
We describe compiler and runtime support to help users automate testing high performance numerical programs.
In the most general approach, the programmer adds API calls or compiler directives to the program where intermediate results should be saved or compared.
Starting with a known correct program, a sequence of intermediate \emph{golden results} will be saved to a file.
In subsequent test runs, the same API calls or directives are used to compare the intermediate results in the running program against the \emph{golden results} previously saved.
We also describe the special case of an OpenACC program targeting GPUs, where the existing data directives and clauses are used to control the compare.
The compiler generates code to run each computational kernel on both the host and the GPU, and compares the results after each kernel execution.

\dots
\keywords{Program testing}
\end{abstract}
%



\section{Introduction}

There are several unique aspects of testing numerical applications used in high performance computing.
Since an important goal is performance, programmers will often try new things for high performance.
They may enable a new compiler optimization, replace a library with an optimized version, try a different algorithm, use OpenMP or OpenACC to run loops in parallel, retarget key loops to run on an accelerator such as a GPU, or test a CPU from a different vendor.
In each of these, it's important to test both whether the performance improves and whether the results are the identical or are only different within acceptable tolerances.
If there are differences, it is important to be identify where the computations start to diverge.
We describe compiler and runtime support to help users automate this testing.
In the most general approach, using API calls or compiler directives, we describe a scheme to save a sequence of intermediate results from a known correct run.
In subsequent test runs, the same API calls or directives are used to compare the running program against the \emph{golden results} previously saved.
In 


\section{Testing Numerical Applications}

The general problem is to test whether changes to a numerical application generate different answers.
Since these are typically floating point applications, the meaning of \emph{different} depends on the datatypes being used and the accuracy being tested.
Until the universal adoption of IEEE floating point arithmetic\cite{goldberg.cs.91}, different vendors had very different floating point representations and rounding rules.
With the same floating point representation, many programmers now expect that moving a program to another computer should produce exactly the same answer.
However, different compilers and libraries (even on the same processor) can give different results, for many well-known reasons:
\begin{itemize}
\item Different optimizations (changing a/5.0 into a*0.2, or using fused multiply-add instructions in different places)
\item Different transcendentals (different implementations of exp, sin, sqrt)
\item For parallel programs, different order of operations, specifically for reductions (sums).
\end{itemize}
A good testing scheme must test for significant different, but allow for insignificant differences, as long as the final result is itself correct.

The test should find whether there are errors, but also identify where the errors are introduced.
Thus, the testing process should compare intermediate results as well as the final result.
The testing procedure also should test that the version of the program being tested is following the same sequence of steps as the original program.
If an intermediate condition goes down a different path, that should be considered an error.

Finally, there is the problem of what to do when an error is found.
It could be treated the same as a floating point exception, giving an error message and terminating the program.
If the program starts taking a different execution path, there is no way to continue to test for errors, so termination is probably the only option.
If the error is some value out of tolerance, another option is to continue, to perhaps identify sources of errors.
One way to do this is to continue until the number of errors reaches an error limit.
Another way is to replace the erroneous values with the known correct values, which would have the advantage of knowing that subsequent errors don't arise from the errors already noted and corrected.

Our mechanism is intended to support numerical applications which compute a number of intermediate values that should be tested for correctness.
This may be intermediate values in a time-step loop, for instance.
When creating the \emph{golden results} file, the mechanism will save identifying information, such as file and function name, line number, variable name, its datatype, the number of values and the values themselves.
When comparing test results against the \emph{golden results} file, the mechanism compares that the program is following the same sequence of steps, and that the values lie within the tolerance.
The user can control a global tolerance, and can override the tolerance for each set of values tested.
The user can also control how much output to generate, and how to proceed after an error is found.

There are testing scenarios that we do not consider here.
For instance, a programmer may want to test whether the program would still be correct in lower precision, or whether parts of the program could use lower precision, to save on memory or time.
We are specifically concerned about cases where a given program is being parallelized, ported, or modified, and the user wants to know that the answers will be preserved.


%


\section{Obstacles for Unit Tests in HPC}

\section {What Problems are addressed}
\subsection{comparing results between a known good version (golden) and a version being tested}
  - deciding what values to save / compare
  - deciding when to save / compare values
  - deciding how to do the compare, esp. for floating point
\subsection{Usage Scenarios}
  - user driven, save golden values to a file, compare test version against those
    using API calls or directives
    user decides what values to save/compare, when to compare
  - automatic online compare, compare CPU vs GPU computations
    can either compare after each kernel launch, or
    compare all values present on device against golden host values


\section{Implementation Details}
   compare options - tolerance, IEEE comparisons
   where in the runtime it is implemented (?)
   challenges: Fortran vs C, gcc-specific header file functionality
   saving a golden file, file block headers to compare execution sequence

\section{Experimentation}
   cost overhead of compare
   cost overhead of autocompare
   what kinds of differences we find

\section{Other Uses of a Compare Feature}
   compare different algorithms
   compare compiler optimizations
   compare different hosts



\begin{figure}
\vspace{2.5cm}
\caption{This is the caption of the figure displaying a white eagle and
a white horse on a snow field}
\end{figure}


\begin{table}
\caption{Environment Variables }
\begin{center}
\begin{tabular}{r@{\quad}ll}
\hline
\multicolumn{1}{l}{\rule{0pt}{12pt} export\_PGICOMPARE = […, …, …] 
}&\multicolumn{2}{l}{  Description }\\[2pt]
\hline\rule{0pt}{12pt}
FILE= & Name of data file & \\
CREATE   &   Specifies that this is the run that will produce the reference results& \\
COMPARE   &   Specifies that the current run will be compared with a reference file& \\
VERBOSE   & Outputs all details of comparison& \\
PATCH   &   Patch with correct values& \\
STOP   &   Stop at first difference & \\
SKIP=    & Skip differences or don’t compare variable & \\
REPORT=    & Report first differences & \\
TOL [ABS, REL, ULP, IEEE]  & Specifies the desired tolerance type &\\[2pt]
\hline
\end{tabular}
\end{center}
\end{table}


\bibliographystyle{IEEEtran}
\bibliography{main}



\end{document}
