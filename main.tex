% This is based on the LLNCS.DEM the demonstration file of
% the LaTeX macro package from Springer-Verlag
% for Lecture Notes in Computer Science,
% version 2.4 for LaTeX2e as of 16. April 2010
%
% See http://www.springer.com/computer/lncs/lncs+authors?SGWID=0-40209-0-0-0
% for the full guidelines.
%
\documentclass{llncs}

\begin{document}

\title{Compiler Approach for Correctness Assurance of High Performance Computing Applications}
%
\titlerunning{Hamiltonian Mechanics}  % abbreviated title (for running head)
%                                     also used for the TOC unless
%                                     \toctitle is used
%
\author{Khalid Ahmad\inst{1} \and Michael Wolfe\inst{2}}
%
\authorrunning{Khalid Ahmad et al.} % abbreviated author list (for running head)
%
%%%% list of authors for the TOC (use if author list has to be modified)
\tocauthor{Khalid Ahmad, Roger Temam}
%
\institute{University of Utah,
, Salt Lake City, Utah, USA\\
\email{Khalid@cs.utah.edu}
\and
PGI Compilers and tools; an NVIDIA brand, Portland, OR, USA}

\maketitle              % typeset the title of the contribution

\begin{abstract}
The abstract should summarize the contents of the paper
using at least 70 and at most 150 words. It will be set in 9-point
font size and be inset 1.0 cm from the right and left margins.
There will be two blank lines before and after the Abstract. \dots
\keywords{computational geometry, graph theory, Hamilton cycles}
\end{abstract}
%



\section{Introduction}
%


\section{Obstacles for Unit Tests in HPC}

\section {What Problems are addressed}
\subsection{comparing results between a known good version (golden) and a version being tested}
  - deciding what values to save / compare
  - deciding when to save / compare values
  - deciding how to do the compare, esp. for floating point
\subsection{Usage Scenarios}
  - user driven, save golden values to a file, compare test version against those
    using API calls or directives
    user decides what values to save/compare, when to compare
  - automatic online compare, compare CPU vs GPU computations
    can either compare after each kernel launch, or
    compare all values present on device against golden host values


\section{Implementation Details}
   compare options - tolerance, IEEE comparisons
   where in the runtime it is implemented (?)
   challenges: Fortran vs C, gcc-specific header file functionality
   saving a golden file, file block headers to compare execution sequence

\section{Experimentation}
   cost overhead of compare
   cost overhead of autocompare
   what kinds of differences we find

\section{Other Uses of a Compare Feature}
   compare different algorithms
   compare compiler optimizations
   compare different hosts



\begin{figure}
\vspace{2.5cm}
\caption{This is the caption of the figure displaying a white eagle and
a white horse on a snow field}
\end{figure}


\begin{table}
\caption{Environment Variables }
\begin{center}
\begin{tabular}{r@{\quad}ll}
\hline
\multicolumn{1}{l}{\rule{0pt}{12pt} export\_PGICOMPARE = […, …, …] 
}&\multicolumn{2}{l}{  Description }\\[2pt]
\hline\rule{0pt}{12pt}
FILE= & Name of data file & \\
CREATE   &   Specifies that this is the run that will produce the reference results& \\
COMPARE   &   Specifies that the current run will be compared with a reference file& \\
VERBOSE   & Outputs all details of comparison& \\
PATCH   &   Patch with correct values& \\
STOP   &   Stop at first difference & \\
SKIP=    & Skip differences or don’t compare variable & \\
REPORT=    & Report first differences & \\
TOL [ABS, REL, ULP, IEEE]  & Specifies the desired tolerance type &\\[2pt]
\hline
\end{tabular}
\end{center}
\end{table}




%
%
% ---- Bibliography ----
%
\begin{thebibliography}{5}
%
\bibitem {clar:eke}
Clarke, F., Ekeland, I.:
Nonlinear oscillations and
boundary-value problems for Hamiltonian systems.
Arch. Rat. Mech. Anal. 78, 315--333 (1982)

\bibitem {clar:eke:2}
Clarke, F., Ekeland, I.:
Solutions p\'{e}riodiques, du
p\'{e}riode donn\'{e}e, des \'{e}quations hamiltoniennes.
Note CRAS Paris 287, 1013--1015 (1978)

\bibitem {mich:tar}
Michalek, R., Tarantello, G.:
Subharmonic solutions with prescribed minimal
period for nonautonomous Hamiltonian systems.
J. Diff. Eq. 72, 28--55 (1988)

\bibitem {tar}
Tarantello, G.:
Subharmonic solutions for Hamiltonian
systems via a $\bbbz_{p}$ pseudoindex theory.
Annali di Matematica Pura (to appear)

\bibitem {rab}
Rabinowitz, P.:
On subharmonic solutions of a Hamiltonian system.
Comm. Pure Appl. Math. 33, 609--633 (1980)

\end{thebibliography}

\end{document}
