% This is based on the LLNCS.DEM the demonstration file of
% the LaTeX macro package from Springer-Verlag
% for Lecture Notes in Computer Science,
% version 2.4 for LaTeX2e as of 16. April 2010
%
% See http://www.springer.com/computer/lncs/lncs+authors?SGWID=0-40209-0-0-0
% for the full guidelines.
%
\documentclass{llncs}

\begin{document}

\title{Compiler Support for Testing Numerical HPC Applications}
%
\titlerunning{Hamiltonian Mechanics}  % abbreviated title (for running head)
%                                     also used for the TOC unless
%                                     \toctitle is used
%
\author{Khalid Ahmad\inst{1} \and Michael Wolfe\inst{2}}
%
\authorrunning{Khalid Ahmad et al.} % abbreviated author list (for running head)
%
%%%% list of authors for the TOC (use if author list has to be modified)
\tocauthor{Khalid Ahmad, Roger Temam}
%
\institute{University of Utah,
, Salt Lake City, Utah, USA\\
\email{Khalid@cs.utah.edu}
\and
PGI Compilers and tools; an NVIDIA brand, Portland, OR, USA}

\maketitle              % typeset the title of the contribution

\begin{abstract}
We describe compiler and runtime support to help users automate testing high performance numerical programs.
In the most general approach, the programmer adds API calls or compiler directives to the program where intermediate results should be saved or compared.
Starting with a known correct program, a sequence of intermediate \emph{golden results} will be saved to a file.
In subsequent test runs, the same API calls or directives are used to compare the intermediate results in the running program against the \emph{golden results} previously saved.
We also describe the special case of an OpenACC program targeting GPUs, where the existing data directives and clauses are used to control the compare.
The compiler generates code to run each computational kernel on both the host and the GPU, and compares the results after each kernel execution.

\dots
\keywords{Program testing}
\end{abstract}
%



\section{Introduction}

There are several unique aspects of testing numerical applications used in high performance computing.
Since an important goal is performance, programmers will often try new things for high performance.
They may enable a new compiler optimization, replace a library with an optimized version, try a different algorithm, use OpenMP or OpenACC to run loops in parallel, retarget key loops to run on an accelerator such as a GPU, or test a CPU from a different vendor.
In each case, it's important to test both whether the performance improves and whether the results are the identical or differ only within acceptable tolerances.
If there are differences, it is important to be identify where the computations start to diverge.

We have developed compiler and runtime support to help users automate this testing, which we call PAST (Program Assisted Software Testing).
In the most general approach, the program saves a sequence of intermediate results from a known correct run in a \emph{golden results} file, using API calls or compiler directives.
In subsequent test runs, the same API calls or directives compare the running program against the \emph{golden results} previously saved.
The comparisons are type-specific and allow for different kinds of tolerances.
When there are unacceptable differences, the user can select the desired behavior, which can range from returning the number of differences to the running program to printing the detail of each of the differences.
The user can also select how to proceed, which can be to stop the program at the first difference or after $n$ differences, or to replace the bad results with the known good values and continue, hoping to find multiple errors in a single run.
This \emph{golden compare} feature requires running the program at least twice, once to collect the \emph{golden results} file and a second time (or more times) to compare to different test cases.

We have also developed support for the special case of an OpenACC program that targets GPUs for parallel execution.
When debugging OpenACC programs on a GPU, it can be a challenge to determine at what point the results start to go bad.
The problem may be as benign as the different accumulation order of parallel reductions, or may be due to stale data on the device because of missing data \emph{update} operations.
Our PAST implementation will execute the parallel kernels on the CPU as well as on the GPU in a single run, and then compare the two results.
This \emph{OpenACC autocompare} has all the same features, including setting tolerances, choosing a behavior and how to proceed, as the \emph{golden compare}.

The next section describes the numerical application testing problem and the usage cases we intend to support.
Section 3 describes the details of how to use PAST in our compiler, and some of the implementation details.
Section 4 describes the OpenACC autocompare feature, including how to use it and other details.
Section 5 gives measurements of the overhead of both the general compare and the OpenACC autocompare feature.


%


\section{Obstacles for Unit Tests in HPC}

\section {What Problems are addressed}
\subsection{comparing results between a known good version (golden) and a version being tested}
  - deciding what values to save / compare
  - deciding when to save / compare values
  - deciding how to do the compare, esp. for floating point
\subsection{Usage Scenarios}
  - user driven, save golden values to a file, compare test version against those
    using API calls or directives
    user decides what values to save/compare, when to compare
  - automatic online compare, compare CPU vs GPU computations
    can either compare after each kernel launch, or
    compare all values present on device against golden host values


\section{Implementation Details}
   compare options - tolerance, IEEE comparisons
   where in the runtime it is implemented (?)
   challenges: Fortran vs C, gcc-specific header file functionality
   saving a golden file, file block headers to compare execution sequence

\section{Experimentation}
   cost overhead of compare
   cost overhead of autocompare
   what kinds of differences we find

\section{Other Uses of a Compare Feature}
   compare different algorithms
   compare compiler optimizations
   compare different hosts



\begin{figure}
\vspace{2.5cm}
\caption{This is the caption of the figure displaying a white eagle and
a white horse on a snow field}
\end{figure}


\begin{table}
\caption{Environment Variables }
\begin{center}
\begin{tabular}{r@{\quad}ll}
\hline
\multicolumn{1}{l}{\rule{0pt}{12pt} export\_PGICOMPARE = […, …, …] 
}&\multicolumn{2}{l}{  Description }\\[2pt]
\hline\rule{0pt}{12pt}
FILE= & Name of data file & \\
CREATE   &   Specifies that this is the run that will produce the reference results& \\
COMPARE   &   Specifies that the current run will be compared with a reference file& \\
VERBOSE   & Outputs all details of comparison& \\
PATCH   &   Patch with correct values& \\
STOP   &   Stop at first difference & \\
SKIP=    & Skip differences or don’t compare variable & \\
REPORT=    & Report first differences & \\
TOL [ABS, REL, ULP, IEEE]  & Specifies the desired tolerance type &\\[2pt]
\hline
\end{tabular}
\end{center}
\end{table}




%
%
% ---- Bibliography ----
%
\begin{thebibliography}{5}
%
\bibitem {clar:eke}
Clarke, F., Ekeland, I.:
Nonlinear oscillations and
boundary-value problems for Hamiltonian systems.
Arch. Rat. Mech. Anal. 78, 315--333 (1982)

\bibitem {clar:eke:2}
Clarke, F., Ekeland, I.:
Solutions p\'{e}riodiques, du
p\'{e}riode donn\'{e}e, des \'{e}quations hamiltoniennes.
Note CRAS Paris 287, 1013--1015 (1978)

\bibitem {mich:tar}
Michalek, R., Tarantello, G.:
Subharmonic solutions with prescribed minimal
period for nonautonomous Hamiltonian systems.
J. Diff. Eq. 72, 28--55 (1988)

\bibitem {tar}
Tarantello, G.:
Subharmonic solutions for Hamiltonian
systems via a $\bbbz_{p}$ pseudoindex theory.
Annali di Matematica Pura (to appear)

\bibitem {rab}
Rabinowitz, P.:
On subharmonic solutions of a Hamiltonian system.
Comm. Pure Appl. Math. 33, 609--633 (1980)

\end{thebibliography}

\end{document}
