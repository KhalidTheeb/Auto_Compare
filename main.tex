% This is based on the LLNCS.DEM the demonstration file of
% the LaTeX macro package from Springer-Verlag
% for Lecture Notes in Computer Science,
% version 2.4 for LaTeX2e as of 16. April 2010
%
% See http://www.springer.com/computer/lncs/lncs+authors?SGWID=0-40209-0-0-0
% for the full guidelines.
%
\documentclass{llncs}

% define tabs for otabbing region
\newcommand{\otabs}{\hspace*{8mm}\=\hspace*{8mm}\=\hspace{8mm}\=}
% otabbing for creating examples
\newenvironment{otabbing}[1][\otabs]
	{\begin{tabbing}#1\kill}
	{\end{tabbing}}
% typewriter font bold, like textbt, but with the C '#' before pragma
\newcommand{\textct}[1]{\texttt{\textbf{\#\detokenize{#1}}}}
% typewriter font bold after detokenizing, so we can use underscores
\newcommand{\textbt}[1]{\texttt{\textbf{\detokenize{#1}}}}
% typewriter font bold without detokenizing, so we can use \{ and \}
\newcommand{\textat}[1]{\texttt{\textbf{#1}}}

\begin{document}

\title{Compiler Support for Testing Numerical HPC Applications}
%
\titlerunning{Testing Numerical HPC Applications}  % abbreviated title (for running head)
%                                     also used for the TOC unless
%                                     \toctitle is used
%
\author{Khalid Ahmad\inst{1} \and Michael Wolfe\inst{2}}
%
\authorrunning{Khalid Ahmad et al.} % abbreviated author list (for running head)
%
%%%% list of authors for the TOC (use if author list has to be modified)
\tocauthor{Khalid Ahmad, Michael Wolfe}
%
\institute{University of Utah,
Salt Lake City, UT, USA\\
\email{Khalid@cs.utah.edu}
\and
NVIDIA/PGI, Beaverton, OR, USA}

\maketitle              % typeset the title of the contribution

\begin{abstract}
We describe PAST (Program Assisted Software Testing), which provides compiler and runtime support to help users automate testing high performance numerical programs.
In the most general approach, the programmer adds PAST API calls or compiler directives to the program where intermediate results should be saved or compared.
Starting with a known correct program, a sequence of intermediate \emph{golden results} will be saved to a file.
In subsequent test runs, the same PAST API calls or directives are used to compare the intermediate results in the running program against the \emph{golden results} previously saved.
We also describe the special case of an OpenACC program targeting GPUs, where the existing data directives and clauses are used to control the compare.
The compiler generates code to run each computational kernel on both the host and the GPU, and the runtime compares the results after each kernel execution.

\dots
\keywords{Program testing}
\end{abstract}
%



\section{Introduction}

There are several unique aspects of testing numerical applications used in high performance computing.
Since an important goal is performance, programmers will often try new things for high performance.
They may enable a new compiler optimization, replace a library with an optimized version, try a different algorithm, use OpenMP or OpenACC to run loops in parallel, retarget key loops to run on an accelerator such as a GPU, or test a CPU from a different vendor.
In each case, it's important to test both whether the performance improves and whether the results are the identical or differ only within acceptable tolerances.
If there are differences, it is important to be identify where the computations start to diverge.

We have developed compiler and runtime support to help users automate this testing, which we call PAST (Program Assisted Software Testing).
In the most general approach, the program saves a sequence of intermediate results from a known correct run in a \emph{golden results} file, using API calls or compiler directives.
In subsequent test runs, the same API calls or directives compare the running program against the \emph{golden results} previously saved.
The comparisons are type-specific and allow for different kinds of tolerances.
When there are unacceptable differences, the user can select the desired behavior, which can range from returning the number of differences to the running program to printing the detail of each of the differences.
The user can also select how to proceed, which can be to stop the program at the first difference or after $n$ differences, or to replace the bad results with the known good values and continue, hoping to find multiple errors in a single run.
This \emph{golden compare} feature requires running the program at least twice, once to collect the \emph{golden results} file and a second time (or more times) to compare to different test cases.

We have also developed support for the special case of an OpenACC program that targets GPUs for parallel execution.
When debugging OpenACC programs on a GPU, it can be a challenge to determine at what point the results start to go bad.
The problem may be as benign as the different accumulation order of parallel reductions, or may be due to stale data on the device because of missing data \emph{update} operations.
Our PAST implementation will execute the parallel kernels on the CPU as well as on the GPU in a single run, and then compare the two results.
This \emph{OpenACC autocompare} has all the same features, including setting tolerances, choosing a behavior and how to proceed, as the \emph{golden compare}.

The next section describes the numerical application testing problem and the usage cases we intend to support.
Section 3 describes the details of how to use PAST in our compiler, and some of the implementation details.
Section 4 describes the OpenACC autocompare feature, including how to use it and other details.
Section 5 gives measurements of the overhead of both the general compare and the OpenACC autocompare feature.


\section{Testing Numerical Applications}

The general problem is to test whether changes to a numerical application generate different answers.
Since these are typically floating point applications, the meaning of \emph{different} depends on the datatypes being used and the accuracy being tested.
Until the universal adoption of IEEE floating point arithmetic\cite{goldberg.cs.91}, different vendors had very different floating point representations and rounding rules.
With the same floating point representation, many programmers now expect that moving a program to another computer should produce exactly the same answer.
However, different compilers and libraries (even on the same processor) can give different results, for many well-known reasons:
\begin{itemize}
\item Different optimizations (changing a/5.0 into a*0.2, or using fused multiply-add instructions in different places), or
\item Different transcendentals (different implementations of exp, sin, sqrt), or
\item For parallel programs, different order of operations, specifically for reductions (sums).
\end{itemize}
A good testing scheme must test for significant differences, but allow for insignificant differences, as long as the final result is correct.

The test should find whether there are errors, but also identify where the errors are introduced.
Thus, the testing process should compare intermediate results as well as the final result.
The testing procedure also should test that the program being tested is following the same sequence of steps as the original program.
If an intermediate condition goes down a different path, that should be considered an error.

Finally, there is the problem of what to do when an error is found.
It could be treated the same as a floating point exception, giving an error message and terminating the program.
If the program starts taking a different execution path, there is no way to continue to test for errors, so termination is probably the only option.
If the error is some value out of tolerance, another option is to continue, perhaps allowing identification of more errors.
One way to do this is to continue until the number of errors reaches an error limit.
Another way is to replace the erroneous values with the known correct values, which would have the advantage of knowing that subsequent errors don't arise from the errors already noted and corrected.

Our mechanism is intended to support numerical applications which compute a number of intermediate values that should be tested for correctness.
These may be intermediate values in a time-step loop, for instance.
When creating the \emph{golden results} file, PAST will save identifying information, such as file and function name, line number, variable name, its datatype, the number of values and the values themselves.
When comparing test results against the \emph{golden results} file, PAST tests that the program is following the same sequence of steps, and that the values lie within the tolerance.
The user can control a global tolerance, and can override the tolerance for each set of values tested.
The user can also control how much output to generate, and how to proceed after an error is found.

There are testing scenarios that we do not consider here.
For instance, a programmer may want to test whether the program would still be correct in lower precision, or whether parts of the program could use lower precision, to save on memory or time.
We are specifically concerned about cases where a given program is being parallelized, ported, or modified, and the user wants to know that the answers will be preserved.


\section{Usage and Implementation}

Our compiler supports two ways to use the reference file feature, with runtime calls or with compiler directives.
The directives simplify the usage quite a bit, since the compiler knows things like the datatype, file and function names, and line number, and can pass this information to the compare routines automatically.
In either case, the user introduces runtime calls or compiler directives at the points in the program where intermediate or final results should be compared.
The directives must be enabled with a command line flag, and will be ignored by other compilers.
This allows a user to leave the directives in place, making their insertion part of the continuing maintenance procedure.
The runtime calls can be placed in conditionally compiled regions, or will have to be isolated in a testing harness, or removed before releasing the software.

The directives use Fortran array notation or the same C array section notation used in OpenACC and OpenMP.
In Fortran, the user can insert a directive like:
\begin{otabbing}
\>\textbt{!$pgi compare(a)}
\end{otabbing}
where \textbt{a} is a Fortran simple variable or array.
The user can have two or more names on a single directive, and can use array sections as well:
\begin{otabbing}
\>\textbt{!$pgi compare(a(:),b(1:100,2:20))}
\end{otabbing}
In C++ or C, the user can insert a pragma like:
\begin{otabbing}
\>\textct{pragma pgi compare(x,a[0:n])}
\end{otabbing}
If \textbt{x} is a scalar, a fixed-size array or C VLA, the compiler knows the size of the array; otherwise, the user must use the array section notation to tell the compiler the size of a dynamically allocated array.
The compiler will interpret the directives, essentially replacing them with runtime calls that handle each array, implicitly including the filename, function name, line number, variable or array name, and datatype information, for more precise error reporting.
There are additional clauses that can appear on the directive, to specify the tolerance to use and how to proceed in case of an error.
The directives are enabled with the \textbt{-Mcompare} compiler flag.

The alternative is to use runtime calls.
In Fortran, the user can insert
\begin{otabbing}
\>\textbt{call pgi_compare(a, n, 'real')} \\
\>\textbt{call pgi_compare(b, 20000, 'double precision')}
\end{otabbing}
In C++ or C, the user would insert the function call:
\begin{otabbing}
\>\textbt{pgi_compare(x, 1, "float",__FILE__,__LINE__)} \\
\>\textbt{pgi_compare(a, n, "double",__FILE__,__LINE__)}
\end{otabbing}
This method requires the user to specify the datatype in a string, and won't get the other identifying information automatically, but the compare procedure is otherwise identical.
This method also loses the ability to specify different tolerances and behavior for different arrays.

Whether using directives or runtime calls, the program will save the values to a reference file, or compare the values to those saved in that file.
In the test run, the runtime routine will validate the program sequence by checking that any available filename and line number information matches, as well as that the saved data sequence matches in datatype and number of values.
The compare operation is controlled by the directive clauses and by the \textbt{PGI_COMPARE} environment variable.
\textbt{PGI_COMPARE} must have a sequence of comma-separated options,
shown in the table in Table~\ref{env}.
By default, if the file does not exist, it is created;
if it does exist, it is treated as the reference file against which to compare.
\begin{table}
\begin{center}
\begin{tabular}{ll}
\hline
option & Description \\
% \multicolumn{1}{l}{\rule{0pt}{12pt} \textbt{export PGICOMPARE=option[,option]}
% }&\multicolumn{1}{l}{  Description }\\[2pt]
\hline
\textbt{FILE=}\textit{filename} & Name of reference file \\
\textbt{CREATE}   &   This run creates the reference file \\
\textbt{COMPARE}   &   Compare this run to the reference file \\
\textbt{ABS=}\textit{r} & Use \textit{r} as an absolute tolerance \\
\textbt{REL=}\textit{r} & Use \textit{r} as a relative tolerance \\
\textbt{ULP=}\textit{n} & Allow \textit{n} differences in ULP \\
\textbt{IEEE} & Test IEEE NaNs\\
\textbt{REPORT=}\textit{n} & Report first \textit{n} differences \\
\textbt{SKIP=}\textit{n}    & Skip the first \textit{n} differences \\
\textbt{VERBOSE}   & Outputs all details of comparison \\
\textbt{PATCH}   &   Patch erroneous values with correct values \\
\textbt{STOP}   &   Stop at first difference \\
\hline
\end{tabular}
\end{center}
\caption{Options that can appear in the \textbt{PGI_COMPARE} environment variable.}
\label{env}
\end{table}

When comparing integer values, the tolerance is ignored and the test value must match the reference value exactly.
When comparing floating point values, the test value will be compared using an absolute tolerance, a relative tolerance, compare ignoring the last ULP digits, or any combination of the three.
If no tolerance is specified, the tolerance is zero.
If any tolerance is exceeded, then the comparison fails.
If the test value is a \emph{NaN} but does not match the reference value, the comparison fails as well.
This comparison scheme was copied from what we use in our nightly and weekly quality assurance tests, where we run over 1,000,000 tests every weekend, plus more during shorter nightly runs.
If a comparison fails, the program can ignore some number of errors or report on them, then either stop, continue execution, or patch the bad errors and then continue.

\section{Autocompare with OpenACC}

When debugging OpenACC programs targeting GPUs, we have additional problems as well as an important advantage.
The problems include using two different processors in the same application, and managing data traffic between the system memory and the GPU device memory.
The important advantage is that we have two processors, so we can create the golden values on the CPU while the GPU is executing, and we have separate memory for the CPU and GPU, so we have a place to store the golden values and the test values.
Many of the problems that arise when programming GPUs with OpenACC have to do with managing the separate memories (stale data on the GPU or the CPU), or dealing with all the problems of porting to a new processor while part of the program is still running on the old processor.

OpenACC programs have directives to tell the compiler what loops to run in parallel on the GPU (\emph{compute regions}), and what data to copy to the device and when to update data between device and host (\emph{data constructs}).
Our \emph{OpenACC autocompare} feature uses these directives, and a runtime table that keeps track of the correspondence between host and device data, to select what data to compare.




\textbf{not perfect, the compare point is a sync point, so this won't help find async errors}

\textbf{Needs more text here...but we need more experience as well}

\section{Experiments}

We have measured the overhead of our compare feature to demonstrate its usability.
For the general case, we measure the overhead of the first run, where data values are written to a file, and the test run, where values are compared to the \emph{golden results} file, in both cases measuring execution time compared to the run with no compare overhead.
For the OpenACC autocompare feature, we measure the overhead of the autocompare against the time with no compare overhead.
With OpenACC autocompare, the main overhead is running the program on the CPU as well as the GPU.


We have run our overhead experiments on three of the SPEC ACCEL v1.2 benchmarks.
In each case, the program has an outer time step loop containing the main computation.
For the general compare feature, we inserted a call or calls to \textbt{pgi_compare} inside the loop, to save and compare the state after each time step.
We used the \emph{ref} data set in each case.
We ran the \emph{golden} run on an Intel machine, and \emph{test} runs on the same machine as well as on an IBM POWER 8 machine.
The values shown are:
\begin{itemize}
\item Time (seconds) to run the original program on each machine.
\item Time and overhead (percent slowdown) to run the \emph{golden} run on the Intel machine.
\item Time and overhead to run the \emph{test} run on the each machine.
\item Number of calls to \textbt{pgi_compare}.
\item Size of the \emph{golden results} file (bytes).
\end{itemize}

\begin{table}
\begin{center}
\begin{tabular}{lll}
\hline
Operation/Data Size & 303.ostencil & 357.csp\\
\hline
\textbt{Binary file size (bytes)} & 80,099 & 418,481\\
\textbt{Write data with reporting}   &   0.0 & 0.0\\
\textbt{Reading and comparing with reporting}   &   0.0 & 0.04\\
\textbt{Compare with report} & 0.0 &0.04\\
\textbt{Write data without reporting} & 0.0 & 0.0\\
\textbt{Reading and comparing without reporting} & 0.0 & 0.0\\
\textbt{Compare without report} & 0.0& 0.0\\
\hline
\end{tabular}
\end{center}
\caption{The overhead time incurred due to using pgi-compare on two Spec accel applications }
\label{res1}
\end{table}



For the OpenACC autocompare feature, we ran the same set of programs, comparing the time of the combined \emph{test} run against both the original program on a CPU, and the original program on the same GPU.
The values shown are:
\begin{itemize}
\item Time (seconds) to run the original program on the CPU.
\item Time to run the original program on the GPU.
\item Time to run combined \emph{test} program, and overhead relative to both the CPU and GPU runs.
\item Number of data variables compared.
\item Bytes of data compared.
\end{itemize}

\textbf{Conclusions about the overheads!}

\section{Conclusion}

The closest prior work to this is the FortranTestGenerator\cite{hovy.iwsehpc.16}, which has some of the same goals.
The FortranTestGenerator is designed to create test cases for specific procedures (Fortran subroutines) by capturing the inputs to the subroutine and generating a driver that will recreate the state of the program before the procedure and then invoke the procedure.
As of publication of the article, validation is not yet automated.
Our design can't be used to create test cases for parts of a program in isolation.
In contrast, our method can test the state of the data at one point, or in many points throughout the execution of a program.

Our design has four advantages.
\begin{itemize}
\item The design requires the user to add runtime API calls or directives only where the data should be saved and compared, minimizing the changes required to the program.
\item The use of directives allows the compiler to implicitly add more information, such as the datatype of the data to be compared, variable name, function name, and file location.
The directives can be maintained in the program, only enabling them with a command line option when testing.
\item The same API calls or directives are used for both the \emph{golden} run, creating the \emph{golden result} file, as for the \emph{test} run, comparing to that file.
The behavior of the API calls or directives are controlled by a runtime environment variable.
\item The procedure can find changes in execution paths as well as changes in data.
\item The OpenACC autocompare allows comparing the GPU code against the CPU code without adding any directives or API calls.
OpenACC programs typically identify data that is copied to the device memory, allowing the CPU code to generate the \emph{golden} results in system memory while the GPU generates \emph{test} results in device memory.
At runtime, the data in the two memories is dynamically compared.
\end{itemize}

With low overhead and flexible compare options, 

%


%\section{Obstacles for Unit Tests in HPC}

%\section {What Problems are addressed}
%\subsection{comparing results between a known good version (golden) and a version being tested}
%  - deciding what values to save / compare
%  - deciding when to save / compare values
%  - deciding how to do the compare, esp. for floating point
%\subsection{Usage Scenarios}
%  - user driven, save golden values to a file, compare test version against those
%    using API calls or directives
%    user decides what values to save/compare, when to compare
%  - automatic online compare, compare CPU vs GPU computations
%    can either compare after each kernel launch, or
%    compare all values present on device against golden host values

%\section{Implementation Details}
%   compare options - tolerance, IEEE comparisons
%   where in the runtime it is implemented (?)
%   challenges: Fortran vs C, gcc-specific header file functionality
%   saving a golden file, file block headers to compare execution sequence

%\section{Experimentation}
%   cost overhead of compare
%   cost overhead of autocompare
%   what kinds of differences we find
%
%\section{Other Uses of a Compare Feature}
%   compare different algorithms
%   compare compiler optimizations
%   compare different hosts

\bibliographystyle{IEEEtran}
\bibliography{main}



\end{document}
