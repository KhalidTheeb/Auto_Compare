\section{Introduction}

There are several unique aspects of testing numerical applications used in high performance computing.
Since an important goal is performance, programmers will often try new things for higher performance.
They may enable a new compiler optimization, replace a library with an optimized version, try a different algorithm, use OpenMP or OpenACC to run loops in parallel, retarget key loops to run on an accelerator such as a GPU, or test a CPU from a different vendor.
In each case, it's important to test both whether the performance improves and whether the results are identical or differ only within acceptable tolerances.
If there are differences, it is important to identify where the computations start to diverge.

We have developed compiler and runtime support to help users automate this testing, which we call CAST (Compiler-Assisted Software Testing).
In the most general approach, the program saves a sequence of intermediate results from a known correct run in a reference file, using runtime calls or compiler directives.
In subsequent test runs, the same runtime calls or directives compare the running program against the reference results previously saved.
The comparisons are type-specific and allow for different kinds of tolerances.
When there are unacceptable differences, the user can select the desired behavior, which can range from returning the number of differences to the running program to printing the detail of each of the differences.
The user can also select how to proceed, which can be to stop the program at the first difference or after $n$ differences, or to replace the bad results with the known good values and continue, hoping to find multiple errors in a single run.
This feature requires running the program at least twice, once to create the reference file and a second time (or more times) to compare to different test cases.

The next section describes the numerical application testing problem and the usage cases we intend to support.
Section 3 describes the details of how to use CAST in our compiler, and some of the implementation details.
Section 4 gives measurements of the runtime overhead of CAST, and the final section describes future work.
