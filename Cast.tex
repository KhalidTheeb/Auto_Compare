% This is based on the LLNCS.DEM the demonstration file of
% the LaTeX macro package from Springer-Verlag
% for Lecture Notes in Computer Science,
% version 2.4 for LaTeX2e as of 16. April 2010
%
% See http://www.springer.com/computer/lncs/lncs+authors?SGWID=0-40209-0-0-0
% for the full guidelines.
%
\documentclass[conference]{IEEEtran}
\usepackage{cite}
\usepackage{amsmath,amssymb,amsfonts}
\usepackage{algorithmic}
\usepackage{graphicx}
\usepackage{textcomp}

% define tabs for otabbing region
\newcommand{\otabs}{\hspace*{8mm}\=\hspace*{8mm}\=\hspace{8mm}\=}
% otabbing for creating examples
\newenvironment{otabbing}[1][\otabs]
	{\begin{tabbing}#1\kill}
	{\end{tabbing}}
% typewriter font bold, like textbt, but with the C '#' before pragma
\newcommand{\textct}[1]{\texttt{\textbf{\#\detokenize{#1}}}}
% typewriter font bold after detokenizing, so we can use underscores
\newcommand{\textbt}[1]{\texttt{\textbf{\detokenize{#1}}}}
% typewriter font bold without detokenizing, so we can use \{ and \}
\newcommand{\textat}[1]{\texttt{\textbf{#1}}}

\begin{document}

\title{Compiler Support for Testing Numerical HPC Applications}
%
%\titlerunning{Testing Numerical HPC Applications}  % abbreviated title (for running head)
%                                     also used for the TOC unless
%                                     \toctitle is used
%
\author{
\IEEEauthorblockN{Khalid Ahmad}
\IEEEauthorblockA{\textit{University of Utah}\\
Salt Lake City, Utah, USA \\
khalid@cs.utah.edu}
\and
\IEEEauthorblockN{Michael Wolfe}
\IEEEauthorblockA{\textit{NVIDIA Corp.} \\
Beaverton, Oregon, USA \\
mwolfe@nvidia.com}
}
%
%\authorrunning{Khalid Ahmad et al.} % abbreviated author list (for running head)
%
%%%% list of authors for the TOC (use if author list has to be modified)
%\tocauthor{Khalid Ahmad, Michael Wolfe}
%
\maketitle              % typeset the title of the contribution

\begin{abstract}
We describe CAST (Compiler-Assisted Software Testing), a feature in our compiler and runtime to help users automate testing high performance numerical programs.
The programmer adds CAST runtime calls or compiler directives to the program where intermediate results should be saved or compared.
Starting with a known correct program, a sequence of intermediate reference results will be saved to a file.
In subsequent test runs, the same CAST runtime calls or directives are used to compare the intermediate results in the running program against the reference results previously saved.
\end{abstract}
\begin{IEEEkeywords}
Program testing
\end{IEEEkeywords}
%



\section{Introduction}

There are several unique aspects of testing numerical applications used in high performance computing.
Since an important goal is performance, programmers will often try new things for higher performance.
They may enable a new compiler optimization, replace a library with an optimized version, try a different algorithm, use OpenMP or OpenACC to run loops in parallel, retarget key loops to run on an accelerator such as a GPU, or test a CPU from a different vendor.
In each case, it's important to test both whether the performance improves and whether the results are identical or differ only within acceptable tolerances.
If there are differences, it is important to identify where the computations start to diverge.

We have developed compiler and runtime support to help users automate this testing, which we call CAST (Compiler-Assisted Software Testing).
In the most general approach, the program saves a sequence of intermediate results from a known correct run in a reference file, using runtime calls or compiler directives.
In subsequent test runs, the same runtime calls or directives compare the running program against the reference results previously saved.
The comparisons are type-specific and allow for different kinds of tolerances.
When there are unacceptable differences, the user can select the desired behavior, which can range from returning the number of differences to the running program to printing the detail of each of the differences.
The user can also select how to proceed, which can be to stop the program at the first difference or after $n$ differences, or to replace the bad results with the known good values and continue, hoping to find multiple errors in a single run.
This feature requires running the program at least twice, once to create the reference file and a second time (or more times) to compare to different test cases.

The next section describes the numerical application testing problem and the usage cases we intend to support.
Section 3 describes the details of how to use CAST in our compiler, and some of the implementation details.
Section 4 gives measurements of the runtime overhead of CAST, and the final section describes future work.

\input{CastProblem}
\section{Usage and Implementation}

Our compiler supports two ways to use CAST, with runtime calls or with compiler directives.
The directives simplify the usage quite a bit, since the compiler knows things like the datatype, file and function names, and line number, and can pass this information to the compare routines automatically.
In either case, the user introduces runtime calls or compiler directives at the points in the program where intermediate or final results should be compared.
The directives must be enabled with a command line flag, and will be ignored by other compilers.
This allows a user to leave the directives in place, making their insertion part of the continuing maintenance procedure.
The runtime calls can be placed in conditionally compiled regions, or will have to be isolated in a testing harness, or removed before releasing the software.

The directives use Fortran array notation or the same C array section notation familiar to OpenACC and OpenMP users.
In Fortran, the user can insert a directive like:
\begin{otabbing}
\>\textbt{!$pgi compare(a)}
\end{otabbing}
where \textbt{a} is a Fortran simple variable or array.
The user can have two or more names on a single directive, and can use array sections as well:
\begin{otabbing}
\>\textbt{!$pgi compare(a(:),b(1:100,2:20))}
\end{otabbing}
In C++ or C, the user can insert a pragma like:
\begin{otabbing}
\>\textct{pragma pgi compare(x,a[0:n])}
\end{otabbing}
If \textbt{x} is a scalar, a fixed-size array or C VLA, the compiler knows its size; the user must use array section notation to tell the compiler the size of dynamically allocated arrays.
The compiler will interpret the directives, essentially replacing them with runtime calls that handle each array, implicitly including the filename, function name, line number, variable or array name, and datatype information, for more precise error reporting.
There are additional clauses that can appear on the directive, to specify the tolerance to use and how to proceed in case of an error.
The directives are enabled with a compiler flag.

The alternative is to use runtime calls.
In Fortran, the user can insert
\begin{otabbing}
\>\textbt{call pgi_compare(a, 'real', n, 'a', 1)} \\
\>\textbt{call pgi_compare(b, 'double precision', 20000, 'b', 2)}
\end{otabbing}
In C++ or C, the user can insert the function call:
\begin{otabbing}
\>\textbt{pgi_compare(x, "float", 1, __FILE__,__LINE__)} \\
\>\textbt{pgi_compare(a, "double", n, "a", 2)}
\end{otabbing}
The programming must specify the datatype, and the last two arguments are a string and an integer, used to find the location where an unacceptable difference occurs.
This method also loses the ability to specify different tolerances and behavior for different arrays.
Otherwise, the compare procedure is otherwise.

Whether using directives or runtime calls, the program will save the values to a reference file, or compare the values to those saved in that file.
In the test run, the runtime routine will validate the program sequence by checking that any available location identification information matches, as well as that the saved data sequence matches in datatype and number of values.
The compare operation is controlled by the directive clauses and by the \textbt{PGI_COMPARE} environment variable.
\textbt{PGI_COMPARE} contains a list of comma-separated options,
shown in Table~\ref{env}.
By default, if the file does not exist, it is created;
if it does exist, it is treated as the reference file against which to compare.
\begin{table}
\begin{center}
\begin{tabular}{ll}
\hline
option & Description \\
% \multicolumn{1}{l}{\rule{0pt}{12pt} \textbt{export PGICOMPARE=option[,option]}
% }&\multicolumn{1}{l}{  Description }\\[2pt]
\hline
\textbt{FILE=}\textit{filename} & Name of reference file \\
\textbt{CREATE}   &   This run creates the reference file \\
\textbt{COMPARE}   &   Compare this run to the reference file \\
\textbt{ABS=}\textit{r} & Use \textit{r} as an absolute tolerance \\
\textbt{REL=}\textit{r} & Use \textit{r} as a relative tolerance \\
\textbt{ULP=}\textit{n} & Allow \textit{n} differences in ULP \\
\textbt{IEEE} & Test IEEE NaNs\\
\textbt{REPORT=}\textit{n} & Report first \textit{n} differences \\
\textbt{SKIP=}\textit{n}    & Skip the first \textit{n} differences \\
\textbt{VERBOSE}   & Outputs all details of comparison \\
\textbt{PATCH}   &   Patch erroneous values with correct values \\
\textbt{STOP}   &   Stop at first difference \\
\hline
\end{tabular}
\end{center}
\caption{Options that can appear in the \textbt{PGI_COMPARE} environment variable.}
\label{env}
\end{table}

When comparing integer values, the tolerance is ignored and the test value must match the reference value exactly.
When comparing floating point values, the test value will be compared using an absolute tolerance, a relative tolerance, compare ignoring the last ULP digits, or any combination of the three.
If no tolerance is specified, the tolerance is zero.
If any tolerance is exceeded, then the comparison fails.
If the test value is a \emph{NaN} but does not match the reference value, the comparison fails as well.
This comparison scheme was copied from what we use in our nightly and weekly quality assurance tests, where we run over 1,000,000 tests every weekend, plus more during shorter nightly runs.
If a comparison fails, the program can ignore some number of errors or report on them, then either stop, continue execution, or patch the bad errors and then continue.

\input{CastExp}
\section{Conclusion}

The closest prior work to this is the FortranTestGenerator\cite{hovy.iwsehpc.16}, which has some of the same goals.
The FortranTestGenerator is designed to create test cases for specific procedures (Fortran subroutines) by capturing the inputs to the subroutine and generating a driver that will recreate the state of the program before the procedure and then invoke the procedure.
As of publication of the article, validation is not yet automated.
Our design can't be used to create test cases for parts of a program in isolation.
In contrast, our method can test the state of the data at one point, or in many points throughout the execution of a program.

Our design has four advantages.
\begin{itemize}
\item The design requires the user to add runtime API calls or directives only where the data should be saved and compared, minimizing the changes required to the program.
\item The use of directives allows the compiler to implicitly add more information, such as the datatype of the data to be compared, variable name, function name, and file location.
The directives can be maintained in the program, only enabling them with a command line option when testing.
\item The same API calls or directives are used for both the \emph{golden} run, creating the \emph{golden result} file, as for the \emph{test} run, comparing to that file.
The behavior of the API calls or directives are controlled by a runtime environment variable.
\item The procedure can find changes in execution paths as well as changes in data.
\item The OpenACC autocompare allows comparing the GPU code against the CPU code without adding any directives or API calls.
OpenACC programs typically identify data that is copied to the device memory, allowing the CPU code to generate the \emph{golden} results in system memory while the GPU generates \emph{test} results in device memory.
At runtime, the data in the two memories is dynamically compared.
\end{itemize}

With low overhead and flexible compare options, 

%


%\section{Obstacles for Unit Tests in HPC}

%\section {What Problems are addressed}
%\subsection{comparing results between a known good version (golden) and a version being tested}
%  - deciding what values to save / compare
%  - deciding when to save / compare values
%  - deciding how to do the compare, esp. for floating point
%\subsection{Usage Scenarios}
%  - user driven, save golden values to a file, compare test version against those
%    using API calls or directives
%    user decides what values to save/compare, when to compare
%  - automatic online compare, compare CPU vs GPU computations
%    can either compare after each kernel launch, or
%    compare all values present on device against golden host values

%\section{Implementation Details}
%   compare options - tolerance, IEEE comparisons
%   where in the runtime it is implemented (?)
%   challenges: Fortran vs C, gcc-specific header file functionality
%   saving a golden file, file block headers to compare execution sequence

%\section{Experimentation}
%   cost overhead of compare
%   cost overhead of autocompare
%   what kinds of differences we find
%
%\section{Other Uses of a Compare Feature}
%   compare different algorithms
%   compare compiler optimizations
%   compare different hosts

\bibliographystyle{IEEEtran}
\bibliography{main}



\end{document}
